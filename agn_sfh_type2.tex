\documentclass[useAMS,usenatbib]{mn2e}
\usepackage{footnote}
\usepackage{graphicx}
\usepackage{amsmath}
\usepackage{natbib}
\usepackage{array}
\usepackage{color}
\usepackage{url}
\usepackage{multirow}
\voffset=-0.5in
\pdfminorversion=5
\setlength{\parsep}{0pt}
\setlength{\partopsep}{0pt}

\definecolor{nc}{rgb}{1,0,0}
\def\changed    {\color{nc} }

\def\starpy ~{\textsc{starpy}}

\begin{document}
\title[Quenching Histories of AGN Host Galaxies]{Galaxy Zoo: Evidence for rapid, recent quenching across a population of AGN host galaxies}
\author[Smethurst et al. 2015]{R. ~J. ~Smethurst,$^{1}$ C. ~J. ~Lintott,$^{1}$ B. ~D. ~Simmons,$^{1}$ K. ~Schawinski,$^{2}$ \newauthor S. ~P. ~Bamford,$^{3}$  C. ~N. ~Cardamone,$^{4}$ S. ~J. ~Kruk,$^{1}$ K. ~L. ~Masters,$^{5}$ \newauthor C. ~M. ~Urry,$^{6}$  K. ~W. ~Willett,$^{7}$ O. ~I. ~Wong$^{8}$ \footnotemark[1]
\\ $^1$ Oxford Astrophysics, Department of Physics, University of Oxford, Denys Wilkinson Building, Keble Road, Oxford, OX1 3RH, UK 
\\ $^2$ Institute for Astronomy, Department of Physics, ETH Zurich, Wolfgang-Pauli Strasse 27, CH-8093 Z\"urich, Switzerland
\\ $^3$ School of Physics and Astronomy, The University of Nottingham, University Park, Nottingham, NG7 2RD, UK
\\ $^4$ Math \& Science Department, Wheelock College, 200 The Riverway, Boston, MA 02215, USA
\\ $^5$ Institute of Cosmology and Gravitation, University of Portsmouth, Dennis Sciama Building, Barnaby Road, Portsmouth, PO1 3FX, UK 
\\ $^6$ Department of Physics and Yale Center for Astronomy and Astrophysics, Yale University, PO Box 208121, New Haven, CT 06520-8121, USA
\\ $^7$ School of Physics and Astronomy, University of Minnesota, 116 Church St SE, Minneapolis, MN 55455, USA
\\ $^8$ International Centre for Radio Astronomy Research, UWA, 35 Stirling Highway, Crawley, WA 6009, Australia
}

\maketitle

\begin{abstract}
%We present the results of the first observational population study of the effects of AGN on the star formation history (SFH) of their host galaxies. A new Bayesian software \starpy ~~allows for the investigation of the SFH as two parameters $[t_q, \tau]$ given an observed optical and NUV colour of a galaxy. The output provides a 2D likelihood distribution in the parameter space which is combined across the populations of AGN host galaxies and compared with that for inactive galaxies. We find that galaxies currently hosting an AGN have undergone very recent rapid quenching across all galaxy masses. This is not seen in those galaxies not currently hosting an AGN. Downsizing, whilst apparent for the inactive galaxies is a secondary effect to that of the impact of the AGN on the SFH of their host galaxies particularly for lower mass galaxies. We speculate that this can all be attributed to the gas reservoirs available for both star formation and feeding the black hole in the evolutionary lifetime of the galaxy.\footnotemark[1]

%BASIC INTRO

%A black hole resides at the centre of every galaxy in the Universe wherein it co-evolves with it's host. Despite its relatively small size, an active black hole (active galactic nucleus; AGN) can have an impact on the evolution of its host galaxy through the feedback of energy into the galaxy system.    
%MORE DETAILED
%The largest fraction of AGN are commonly found in the `green valley', suggesting a relation between the AGN and the suppression (quench) of star formation, causing the galaxy to transition from the star forming `blue cloud' to the dead `red sequence'.
%GENERAL PROBLEM
%Though observational evidence has been found for both positive and negative feedback caused by an AGN, the effect on the galaxy wide star formation history (SFH) due to the presence of an AGN has not been studied.
%MAIN RESULT
%Here 
We present a population study of the star formation history of $1,244$ Type 2 AGN host galaxies, compared to $123,243$ inactive galaxies using a Bayesian method. We find evidence for the Type 2 AGN host galaxies having undergone a recent (within $2$ Gyr) and rapid drop in their star formation rate. AGN feedback is therefore important at least for this population of galaxies.
%RESULT COMPARISON
This result is not seen for the inactive galaxies whose star formation histories are dominated by the effects of downsizing at earlier epochs, a secondary effect for the AGN host galaxies. We show that  histories of rapid quenching cannot account fully for the quenching of all the star formation in a galaxy's lifetime across the population of AGN host galaxies, and that histories of slower quenching, attributed to secular (non-violent) evolution, are also key in their evolution.
%RESULTS IN GENERAL CONTEXT
This is in agreement with recent results showing both merger-driven and non-merger processes are contributing to the co-evolution of galaxies and supermassive black holes. The availability of gas in the reservoirs of a galaxy, and its ability to be replenished, appear to be the key drivers behind this co-evolution.
%BROAD PERSPECTIVE

\end{abstract}

\footnotetext[1]{This investigation has been made possible by the participation of over 350,000 users in the Galaxy Zoo project. Their contributions are acknowledged at \url{http://authors.galaxyzoo.org}}

\section{Introduction}

%Two open issues in modern astrophysics are: (i) how do galaxies and their central black holes co-evolve and (ii) what are the effects, if any, of AGN feedback. There is an obvious relationship between the mass of a central black hole and its host galaxy, commonly referred to as the Maggorian relationship \citep{Mag98, MH03, HR04}. 
The nature of the observed co-evolution of galaxies and their central supermassive black holes \citep{Mag98, MH03, HR04} and the effects of AGN feedback on galaxies are two of the most important open issues in galaxy evolution. AGN feedback was first suggested as a mechanism for regulating star formation in simulations \citep{SR98, Croton06, Bower06, Somer08} and some indirect evidence has been observed for both positive and negative feedback in various systems (see the comprehensive review from \citealt{Fab06}). 

The strongest observational evidence for AGN feedback in a population is that the largest fraction of AGN are found in the green valley \citep{CB08, Hickox09, Sch2010}, suggesting some link between AGN activity and  the process of quenching which moves a galaxy from the blue cloud to the red sequence. However, concrete statistical evidence for the effect of AGN feedback on the host galaxy population has so far been elusive.

Here we present a large observational population study of the quenching of Type 2 AGN the host galaxies identified by line diagnostics. We use a new Bayesian method \citep{Sme2015} to effectively model the SFH of a galaxy with two parameters, time of quenching, $t_q$, and exponential rate, $\tau$, given the observed near ultra-violet (NUV) and optical colours. {\changed This is complementary to the work of  \citet{Martin07} who using a similar model, estimated mass fluxes across the green valley; this work demonstrates a significant improvement over this previous study through the use of a Bayesian-MCMC methodology.} Through this we aim to determine the following: (i) Are galaxies currently hosting an AGN undergoing quenching? (ii) If so, when and at what rate does this quenching occur? (iii) Is this quenching occurring at different times and rates compared to a control sample of inactive galaxies?

The zero points of all magnitudes are in the AB system and where necessary, we adopt the WMAP Seven-Year Cosmology (Jarosik et al. 2011) with $(\Omega_m , ~\Omega_\Lambda , ~h) = (0.26, 0.73, 0.71)$.

%\begin{figure*}
%\includegraphics[width=\textwidth]{fig1.pdf}
%\caption{Randomly selected SDSS \emph{gri} composite images from the sample of $1,244$ Type 2 AGN in a redshift range $0.04 < z < 0.05$.  The galaxies are ordered from least to most featured according to their debiased `disc or featured' vote fraction, $p_d$ (see \citealt{GZ2}). The scale for each image is $0.099~\rm{arcsec/pixel}$.}
%\label{mosaic}
%\end{figure*}

\section{Data \& Methods}

In this investigation we use visual classifications of galaxy morphologies from the Galaxy Zoo 2\footnote{\url{http://zoo2.galaxyzoo.org/}} (GZ2) citizen science project \citep{GZ2}, which obtains multiple independent classifications for each optical image. The full question tree for an image is shown in Figure 1 of \citeauthor{GZ2}  The GZ2 project used $304, 022$ images from the Sloan Digital Sky Survey Data Release 7 (SDSS; \citealt{York2000, Abazajian09}) all classified by \emph{at least} 17 independent users, with a mean number of classifications of $\sim42$.

Further to this, we required NUV photometry from the GALEX survey, within which $\sim42\%$ of the GZ2 sample was observed, giving $126, 316$ galaxies total ($0.01 < z < 0.25$). This will be referred to as the \textsc{gz2-galex} sample. The completeness of this sample ($-22 < M_u < -15$) is shown in Figure 2 of \cite{Sme2015}. 

{\changed Observed fluxes are corrected for galactic extinction \citep{Oh11} by applying the \citet*{Cardelli89} law. We also adopt k-corrections to $z = 0.0$ and obtain absolute magnitudes from the NYU-VAGC \citep{Blanton05, Pad08, BR07}.}
%with the $u$-band absolute magnitude against redshift for this sample in comparison to the SDSS data set. 

%Typical Milky Way $L_*$ galaxies with $M_u \sim -20.5$ are still included in the GZ2 subsample out to the highest redshift of $z \sim 0.25$; however dwarf and lower mass galaxies are only detected at the lowest redshifts. 



\subsection{Bayesian SFH Determination}\label{starpy}

\textsc{starpy}\footnote{Publicly available: \url{http://github.com/zooniverse/starpy}} is a \textsc{python} code which allows the user to derive the quenched star formation history (SFH) of a galaxy through a Bayesian Markov Chain Monte Carlo method \citep{Dan}\footnote{\url{http://dan.iel.fm/emcee/}} with the input of the observed $u-r$ and $NUV-u$ colours, a redshift, and the use of the stellar population models of \cite{BC03}. {\changed These models are implemented using solar metallicity and a Chabrier IMF \citep{Chab03}}. The star formation history template is an exponential decline of the SFR and is described by two parameters $[t_q, \tau]$, where $t_q$ is the time at which the onset of quenching begins $\rm{[Gyr]}$ and $\tau$ is the exponential rate at which quenching occurs $\rm{[Gyr]}$. Under the simplifying assumption that all galaxies formed at $t=0$ $\rm{ Gyr}$ with an initial burst of star formation, the SFH can be described as: 
\begin{equation}\label{sfh}
SFR =
\begin{cases}
i_{sfr}(t_q) & \text{if } t < t_q \\
i_{sfr}(t_q) \times exp{\left( \frac{-(t-t_{q})}{\tau}\right)} & \text{if } t > t_q 
\end{cases}
\end{equation}
where $i_{sfr}$ is an initial constant star formation rate dependent on $t_q$ \citep{Sch2014, Sme2015}.  A smaller $\tau$ value corresponds to a rapid quench, whereas a larger $\tau$ value corresponds to a slower quench. {\changed \cite{Martin07} showed that such an exponential SFH leads to the largest UV colour change over time giving an upper limit on the mass flux of galaxies quenching to the red sequence. They concluded that this SFH was the most appropriate when studying such a large population of galaxies in comparison to smooth, burst and complex SFHs. Such an investigation is also planned using \starpy~ in future work.} 

The probabilistic fitting methods to this SFH for an observed galaxy are described in full detail in {\changed Section 3.2 of} \cite{Sme2015} wherein the \starpy ~~code was used to characterise the SFHs of the \textsc{gz2-galex} sample. {\changed We assume a flat prior on all the model parameters and a double gaussian likelihood function for the $u-r$ and $NUV-u$ colours (Equation 2 in \citealt{Sme2015}).} The output of \starpy  ~ is probabilistic in nature and provides the posterior probability distribution across the entirety of the two parameter space for an individual galaxy. {\changed To study the SFH across a given population of galaxies, these individual posterior probability distributions can be combined by summing each binned distribution. We also utilise the GZ debiased vote fractions to obtain separate summed posterior probability distributions for both smooth and disc galaxies; obtained by weighting by either the disc, $p_d$ or smooth, $p_s$ vote fraction of each galaxy when summing. This ensures that the entirety of the population is used, with galaxies with a higher $p_d$ contributing more to the disc weighted than the smooth weighted posterior distribution. This negates the need for a threshold on the GZ vote fractions (e.g. $p_d > 0.8$), unlike in previous studies \citep{Sch2014}.}

%These individual galaxy likelihoods can be combined to visualise the areas of high probability in the model parameter space for a given population of galaxies (e.g. the green valley as in \citealt{Sme2015} and here with active and inactive galaxies). The GZ2 data also provides uniquely powerful continuous measurements of a galaxy�s morphology, therefore we utilise the user vote fractions to obtain separate model likelihood parameter distributions for both smooth and disc galaxies; obtained by weighting by the morphology vote fraction of each galaxy when combined. We stress that this portion of the methodology is a non-Bayesian visualisation of the combined individual galaxy results for each population.


\subsection{AGN Sample}\label{agnsample}

\begin{figure*}
\includegraphics[width=0.94\textwidth]{bpt/GZ2_GALEX_sample_bpt_diagram_SNR_gtr_3_type_1_seyferts_only_better_quality.pdf}
\caption{BPT diagrams for galaxies in the \textsc{gz2-galex} sample (black crosses) with S/N $> 3$ for each emission line. Inequalities defined in: \cite{Kew01} to separate SF galaxies from AGN (dashed lines), \cite{Kauff03b} to separate SF from composite SF-AGN galaxies (solid line) and \cite{Kew06} to separate LINERS and Seyferts (dotted lines). Galaxies are included in the \textsc{agn-host} sample (red circles) if they satisfy all the inequalities to be classified as Seyferts. LINERs are excluded for purity.}
\label{bpt}
\end{figure*}


We selected Type 2 AGN using a BPT diagram \citep{bpt81} using line and continuum strengths for [OIII], [NII], [SII] and [OII] obtained from the MPA-JHU catalogue \citep{Kauff03, Brinch04} for galaxies in the \textsc{gz2-galex} sample. We then required the S/N $> 3$ for each emission line as in \cite{Sch2010}. Those galaxies which satisfied all of the inequalities defined in \cite{Kew01} and \cite{Kauff03b} were selected as Type 2 AGN, giving $1,299$ host galaxies ($\sim10\%$ of the \textsc{gz2-galex} sample). \cite{Sarzi10, RB12} and \cite{Singh13} have all demonstrated that LINERs are not primarily powered by AGN, therefore for purity, we excluded these galaxies from the sample using the definition from \cite{Kew06} ($55$ galaxies total) with no change to the results. These $1,244$ galaxies will be referred to as the \textsc{agn-host} sample. 

Type 2 AGN were used in this analysis as opposed to Type 1 due to their photometric obscuration. Type 1 AGN contaminate their galaxy's photometric measurements which would need to be removed through aperture matching. Due to the requirement for NUV colours from GALEX, in order to be sensitive to any recent star formation, the aperture matching to SDSS becomes a non-trivial task.

\cite{Simmons11} showed that the obscuration of a Type 2 AGN is more efficient in the NUV than in the optical. Residual NUV flux from the AGN can thus be neglected in comparison to that of the galaxy. However there is often some residual optical flux that can affect the measurements of the host galaxy's photometry. We can subtract this central optical AGN flux using the magnitudes provided by SDSS fit to a central point spread function (PSF), however the change in the colours of these galaxies after this correction is negligible, with $\Delta(u-r) \sim 0.09$. We therefore use the uncorrected colours to avoid unnecessary complexity and minimise the propagation of uncertainty from the colours through to the SFHs. Including these corrected colours does not change the results described below.  

Figure \ref{bpt} shows those galaxies in the \textsc{agn-host} sample marked in red and the matched \textsc{gz2-galex} galaxies in black on a BPT diagram.  For the \textsc{agn-host} sample the mean $\log (L[OIII] ~[\rm{erg~s^{-1}}]) \sim 41.3$ and median $\log (L[OIII] ~[\rm{erg~s^{-1}}]) \sim 41.0$, with a range of $\log (L[OIII] ~[\rm{erg~s^{-1}}])$ luminosities of $39.4-43.0$. 

%All of the galaxies satisfying any of the inequalities defined in \cite{Kew01} are removed from the \textsc{gz2-galex} sample along with any of those identified as Type 1 AGN by \citet{Oh15} to produce a largely inactive sample of typical galaxies. These inactive galaxies are used as a control sample to the Type 2 AGN sample and will be referred to as the \textsc{inactive} sample. 

We constructed a sample of inactive galaxies by removing from the \textsc{gz2-galex} sample {\changed all those galaxies above the \citet*{Kauff03b} line separating star forming from active galaxies on the BPT diagram}, as well as sources identified as Type 1 AGN by the presence of broad emission lines \citep{Oh15}. {\changed Each galaxy in the \textsc{agn-host} sample was matched by stellar mass (to within $\pm5\%$) and GZ `smooth' and `disc' vote fractions, $p_s$ and $p_d$ (to within $\pm 0.1$) to at least one, and up to five, inactive galaxies to give $6107$ galaxies.} We refer to this sample as the \textsc{inactive} sample. A Kolmogorov-Smirnov test revealed the redshift distributions of the \textsc{inactive} and \textsc{agn-host} samples are statistically indistinguishable ($D \sim 0.16$, $p \sim 0.88). 

Since this investigation is focussed on whether an AGN can have an impact on the SF of its host galaxy, we must consider whether there is a selection effect present in this identification method. The extent to which SF could be obscured by AGN emission was addressed by \cite{Sch2010}. They showed, through a simple empirical experiment which simulated the addition of an AGN of known luminosity to a star forming galaxy, that BPT-based selection of AGN produces a complete sample, even in the blue cloud, at luminosities of $L[OIII] > 10^{40} \rm{erg~s}^{-1}$. Above this limit we therefore assume we have selected a complete sample of AGN independent of host galaxy SFR.

We also split both the \textsc{agn-host} and \textsc{inactive} samples into low, medium and high mass ranges (see Table \ref{massbins}) to investigate any trends in the SFH with mass. {\changed The mass boundaries were chosen to give roughly equal numbers of inactive galaxies in each bin prior to the mass matching to the \textsc{agn-host} sample.} Masses were calculated using the $(u-r)$ colour and absolute $r$-band magnitude with the method outlined in \cite{Baldry06}.


\section{Results}

%We apply the method outlined in Section \ref{starpy} to obtain a 2-dimensional likelihood distribution across the $[t_q, \tau]$ parameter space. We combine individual galaxy likelihood distributions within the \textsc{agn-host} and \textsc{inactive} galaxy samples, additionally weighting by GZ2 morphologies to produce quenching parameter likelihood distributions for both disc- and smooth-dominated galaxy populations in the three mass bins defined in Table \ref{massbins}. \cite{Sch2014} showed the morphological dependance of quenching histories therefore it is important to distinguish by morphological type. 
%\begin{table}
%\centering
%\caption{Table showing the number of galaxies in each of the three mass bins for both the \textsc{agn-hosts} and \textsc{inactive} galaxy samples and the percentage of the summed PDF across each population found in the rapid, intermediate and slow quenching regimes.}
%\label{massbins}
%\begin{tabular}{c|cc}
%\hline
% Mass Bin                                    & \textsc{agn-hosts} & \textsc{inactive} \\ \hline
%$\log [M_*/M_{\odot}] < 10.25 $           & 165 (13.3\%)         & 41197 (33.4\%)      \\
%$10.25 < \log [M_*/M_{\odot}] < 10.75$ & 630 (50.6\%)         & 46428 (37.7\%)      \\
%$\log [M_*/M_{\odot}] > 10.75$           & 449 (36.1\%)         & 35618  (28.9\%)  \\\hline          
%\end{tabular}
%\end{table}

\begin{table*}
\centering
\caption{Table showing the number of galaxies in each of the three mass bins for both the \textsc{agn-hosts} and \textsc{inactive} galaxy samples and the percentage of the summed PDF across each population found in the rapid, intermediate and slow quenching regimes.}
\label{massbins}
\begin{tabular}{c|c|c|c|c|c|c}
\hline
\textsc{sample}                     & \textsc{mass bin}                                        & \textsc{weighting}                  & $\tau < 1 ~\rm{[Gyr]}$                             & $1 < \tau ~\rm{[Gyr]} < 2 $          & $\tau > 2 ~\rm{[Gyr]}$                               & \textsc{number}                                        \\ \hline \hline
\multirow{6}{*}{AGN-HOSTS} & \multirow{2}{*}{$\log [M_*/M_{\odot}] < 10.25 $}                       & $p_d$     & $60\pm_{5}^{23}$                    & $13\pm_{9}^{9}$                    & $28\pm_{19}^{6}$       & \multirow{2}{*}{$165 (13.3\%)$}                      \\
                           &                                                 & $p_s$     & $69\pm_{6}^{14}$                    & $17\pm_{14}^{6}$                   & $14\pm_{7}^{3}$        &                                                      \\ \cline{2-7} 
                           & \multirow{2}{*}{$10.25 < \log [M_*/M_{\odot}] < 10.75$}                    & $p_d$     & $33\pm_{5}^{3}$                     & $15\pm_{4}^{4}$                    & $51\pm_{7}^{4}$        & \multirow{2}{*}{$630 (50.6\%)$}                      \\
                           &                                                 & $p_s$     & $69\pm_{5}^{4}$                     & $7\pm_{4}^{4}$                     & $26\pm_{9}^{5}$        &                                                      \\ \cline{2-7} 
                           & \multirow{2}{*}{$\log [M_*/M_{\odot}] > 10.75$}                      & $p_d$     & $20\pm_{4}^{5}$ & $25\pm_{5}^{7}$                    & $56\pm_{12}^{8}$       & \multirow{2}{*}{$449 (36.1\%)$}                      \\
                           &                                                 & $p_s$     & $24\pm_{3}^{4}$                     & $26\pm_{6}^{5}$                    & $50\pm_{7}^{7}$        &                                                      \\ \hline \hline
\multirow{6}{*}{INACTIVE}  & \multirow{2}{*}{$\log [M_*/M_{\odot}] < 10.25 $}                       & $p_d$     & $37\pm_{14}^{8}$                    & $39\pm_{6}^{8}$                    & $24\pm_{6}^{8}$        & \multirow{2}{*}{$807 (13.2\%)$}                      \\
                           &                                                 & $p_s$     & $47\pm_{11}^{5}$                    & $36\pm_{5}^{9}$                    & $17\pm_{5}^{4}$        &                                                      \\ \cline{2-7} 
                           & \multirow{2}{*}{$10.25 < \log [M_*/M_{\odot}] < 10.75$}                    & $p_d$     &                                    &                                   &                       & \multirow{2}{*}{$3094 (50.7\%)$}                     \\
                           &                                                 & $p_s$     & \multicolumn{1}{l}{}               & \multicolumn{1}{l}{}              & \multicolumn{1}{l|}{} &                                                      \\ \cline{2-7} 
                           & \multicolumn{1}{l|}{\multirow{2}{*}{$\log [M_*/M_{\odot}] > 10.75$}} & $p_d$     & \multicolumn{1}{l}{}               & \multicolumn{1}{l}{}              & \multicolumn{1}{l|}{} & \multicolumn{1}{l}{\multirow{2}{*}{$2206 (36.1\%)$}} \\
                           & \multicolumn{1}{l|}{}                           & $p_s$      & \multicolumn{1}{l}{}               & \multicolumn{1}{l}{}              & \multicolumn{1}{l|}{} & \multicolumn{1}{l}{}                                 \\ \hline                       
\end{tabular}
\end{table*}

Figures \ref{time} and \ref{rate} show the summed posterior probability distributions for the quenching time, $t_q$ and exponential quenching rate, $\tau$, respectively. In each figure the summed 1-dimensional normalised posterior probability distribution across the given parameter is shown for smooth and disc dominated galaxies across three mass bins for the \textsc{agn-host} and \textsc{inactive} samples. {\changed In Table \ref{massbins} the percentage of the summed posterior distribution in each quenching regime} for rapid ($\tau < 1$ Gyr), intermediate ($1 < \tau ~\rm{[Gyr]} < 2$) and slow ($\tau > 2$ Gyr) quenching timescales, are shown. {\changed Errors on the percentages are calculated from the range of values spanned by $1000$ bootstrap iterations each sampling $90\%$ of the population.}

It is immediately apparent from Figures \ref{time} and \ref{rate} that there is a distinct difference between the posterior probability distributions of \textsc{agn-host} and \textsc{inactive} populations.%For the \textsc{inactive} sample, the likelihood for quenching at later times decreases with increasing mass; for the lower mass galaxies the quenching is roughly constant with time for both morphologies (right panel Figure \ref{time}). 


At all masses, the posterior distribution for the \textsc{agn-host} population across the quenching time $t_q$ parameter (left panels of Figure \ref{time}) is different from that of the inactive galaxies (right panels of Figure \ref{time}). Recent quenching ($t > 11$ Gyr) of \textsc{agn-host} galaxies is the dominant history for low and medium mass galaxies, particularly for the smooth galaxy population. However, this effect is less dominant in higher mass galaxies where quenching at earlier times has significant probability.


The distributions of probability for the quenching rate, $\tau$, in Figure \ref{rate} and Table \ref{massbins} show the dominance of rapid quenching ($\tau < 1$ Gyr) across the \textsc{agn-host} population, particularly for smooth galaxies. With increasing mass the dominant quenching rate becomes slow ($\tau > 2$ Gyr) especially for disc galaxies hosting an AGN. Similar trends in the probability are observed for the \textsc{inactive} population but the overall distribution is very different. 

\begin{figure}
\includegraphics[width=0.485\textwidth]{bpt/quenching_time_histograms_smooth_red_disc_blue_inactive_mass_matched_5pc_sample_with_mode_and_66pc_around_mode.pdf}
\caption{Normalised posterior probability distribution for the quenching time, $t_q$ parameter normalised so that the areas under the curves are equal. \textsc{agn-host} (left) and \textsc{inactive} (right) galaxies are split into low (top), medium (middle) and high (bottom) mass for smooth (red dashed) and disc (blue solid) galaxies. {\changed The mode, along with $\pm1\sigma$, are shown by the solid and dashed lines respectively for each distribution.} A low value of $t_q$ corresponds to the early Universe and a high value to the recent Universe.}
\label{time}
\end{figure}

The posterior probability distribution for the \textsc{agn-host} galaxies therefore shows evidence for the dominance of rapid, recent quenching having occurred across the entire population. This result implies the importance of AGN feedback for the evolution of these galaxies.

\section{Discussion}\label{dis}

The differences between the probability distribution of the \textsc{agn-host} and \textsc{inactive} galaxy populations reveal that an AGN can have a significant effect on the SFH of its host galaxy. Both recent, rapid quenching and early, slow quenching are observed in the probability distribution of the \textsc{agn-host} population. 

{\changed There are minimal morphological differences across the \textsc{agn-host} population between the smooth and disc weighted posterior probability distributions. This is agreement with the conclusions of \citet*{Kauff03b} who found that the structural properties of AGN hosts depend very little on AGN power. }

%As well as the evidence for downsizing in the \textsc{inactive} population (Figure \ref{time}), we can also see its effect on the \textsc{agn-host} population. Although recent quenching is the dominant history, quenching at earlier times has significant likelihood, increasing with mass. These galaxies may have been affected by downsizing earlier in life with the current AGN also having an effect on the SFR through feedback, causing a recent, rapid quench of any residual star formation. 

\begin{figure}
\includegraphics[width=0.485\textwidth]{bpt/quenching_rate_histograms_smooth_red_disc_blue_inactive_mass_matched_5pc_sample_with_mode_and_66pc_around_mode.pdf}
\caption{Likehood distribution for the quenching rate, $\tau$ normalised so that the areas under the curves are equal. \textsc{agn-host} (left) host and \textsc{inactive} (right) galaxies are split into low (top), medium (middle) and high (bottom) mass for smooth (red dashed) and disc (blue solid) galaxies. {\changed The mode, along with $\pm1\sigma$, are shown by the solid and dashed lines respectively for each distribution.} A small (large) value of $\tau$ corresponds to a rapid (slow) quench.}
\label{rate}
\end{figure}


The difference between the \textsc{agn-host} and \textsc{inactive} population probability distributions in Figure \ref{rate} for the rate of quenching, $\tau$, tells a story of gas reservoirs. The distribution of probability for higher mass \textsc{agn-host} galaxies is dominated by slow, early quenching implying another mechanism is responsible for the cessation of star formation in these high mass galaxies prior to the triggering of the current AGN.  This preference for slow evolution timescales follows from the ideas of previously isolated discs evolving slowly by the Kennicutt-Schmidt \citep{Schmidt59, Kennicutt97} law which can then undergo an interaction or merger to reinvigorate star formation, feed the central black hole and trigger an AGN \citep{Varela04, Em15}. These galaxies would need a large enough gas reservoir to fuel both SF throughout their lifetimes and the recent AGN. These high mass galaxies also play host to the most luminous AGN (mean $\log (L[OIII] ~[\rm{erg}~s^{-1}]) \sim 41.6$) and so this SFH challenges the usual explanation for the co-evolution of luminous black holes and their host galaxies driven by merger growth. 



Quenching at early times is also observed for the \textsc{inactive} population, where the probability distribution for the quenching time is roughly constant until recent times where the distribution drops off. {\changed This drop-off occurs at earlier times with increasing mass with a significant lack of probability of quenching at early times for low mass \textsc{inactive} galaxies} (right panels Figure \ref{time}). This is evidence of downsizing across the entire \textsc{inactive} galaxy population whereby stars in massive galaxies form first and quench early \citep{Cowie96, Thomas10}. 

The most massive \textsc{agn-host} galaxies also show a preference for earlier quenching (bottom left panel Figure \ref{time}) occurring at slow rates; we speculate that this is also due to the effects of downsizing rather than being caused by the current AGN. This earlier evolution would first form a slowly `dying' or `dead' galaxy typical of massive elliptical galaxies which can then have a recent infall of gas either through a minor merger, galaxy interaction or environmental change, triggering further star formation and feeding the central black hole, triggering an AGN \citep{Kav14}. In turn this AGN can then quench the recent boost in star formation. This track is similar to the evolution history proposed for blue ellipticals \citep{Kav13, McIntosh14, Haines15}. This SFH would then give rise to the probability distribution seen across the high mass \textsc{agn-host} population for both time and rate parameters.

%If a slowly `dying' or `dead' galaxy has an infall of gas either through a minor merger, galaxy interaction or environmental change, this can trigger further star formation and feed the central black hole, igniting an AGN \citep{Kav14}. In turn this AGN can then quench the recent boost in star formation. This track is similar to the evolution history theorised for blue ellipticals \citep{Kav13, McIntosh14, Haines15}. As for the disc galaxies, this preference for slower evolution timescales follows from the ideas of previously isolated discs evolving slowly by the Kennicutt-Schmidt \citep{Schmidt59, Kennicutt97} law which can then undergo an interaction or merger to reinvigorate star formation and feed the central black hole. 

%\cite{Sme2015} speculate that rapid quenching rates could be attributed to mergers of galaxies, therefore we expect the trend of increasing likelihood for rapid quenching rates with increasing mass for the \textsc{inactive} population, as mergers are thought to be responsible for creating the most massive smooth galaxies \citep{Con03, SdMH05, Hopkins08}. 

%Similarly we also expect the trend of increasing likelihood for slow quenching rates with increasing mass as \cite{Sme2015} attributed slower quenching histories to secular (non-violent) evolution of isolated galaxies \citep{KK04, Cisternas11} which are often lower in mass \citep{Varela04, Bamford09} due to their isolation from other galaxies and therefore any potential gas reservoirs. 
 
%The trends with likelihood of $\tau$ in Figure \ref{rate} are similar for the \textsc{agn-host} and \textsc{inactive} populations but the overall distribution is very different. The likelihood for rapid quenching decreases with increasing mass for both smooth and disc dominated \textsc{agn-host} populations. There is also more dominance of slower quenching timescales in the higher mass AGN population across both morphologies. It appears that the most massive disc galaxies, with the most massive gas reservoirs, evolve with a slow quench of star formation through the Kennicutt-Schmidt law and also have enough gas to feed the central black hole to trigger a current AGN \citep{Varela04, Em15}. This challenges the typical merger-driven theory for the co-evolution of the most luminous black holes and their host galaxies.


These recently triggered AGN in both massive disc and smooth galaxies do not have have the ability to impact the SF across the entirety of a high mass galaxy in a deep gravitational potential \citep{Ish12, Zinn13}. This leads to the lower peak for recent, rapid quenching in the high mass \textsc{agn-host} population for both morphologies. 

Conversely, rapid quenching, possibly caused by the AGN itself through negative feedback, is the most dominant history for low mass \textsc{agn-host} galaxies with lower gravitational potentials which allow gas to be expelled more easily (or heated by the AGN) across the entire galaxy \citep{Torbra09}. 

{\changed Conversely work by \citet{Yesuf14} has shown that AGN are more commonly hosted by post starburst galaxies, with the peak AGN activity appearing significantly after the peak star formation activity by $\geq 200 \pm 100 ~\rm{Myr}$. This implies that AGN may play only a secondary role in quenching after a starburst (possibly merger fueled) has exhausted the bulk of it's gas supply \citep{Croton06, Wild09, Snyder11, Hayward14}. The AGN may therefore be a consequence of another mechanism (e.g. a starburst) which has also caused the rapid, recent quenching observed across the \textsc{agn-host} population.}

However, \cite{Torbra09} model the effects of negative AGN feedback on a typical early type (i.e. smooth) galaxy and find the time between the current galaxy age in their simulation, $t_{gal}$ and the time that the feedback began, $t_{AGN}$, peaks at $t_{gal} - t_{AGN} \sim 0.85 ~\rm{Gyr}$. This agrees with the location of the peak in Figure \ref{time} for low mass galaxies, where the difference between the peak of the probability and the average age of the population (galaxy age is calculated from the redshift by assuming all galaxies form at $t=0$) is $\sim0.83 ~\rm{Gyr}$. This implies that this dominant recent quenching history is caused directly by AGN feedback, as opposed to the AGN being a consequence of an alternative quenching mechanism.

Rapid quenching is particularly dominant for low-to-medium mass smooth galaxies. \cite{Sme2015} suggest that incredibly rapid quenching rates could be attributed to mergers of galaxies in conjunction with AGN feedback, which are thought to be responsible for creating the most massive smooth galaxies \citep{Con03, SdMH05, Hopkins08}. This dominance of rapid quenching across the smooth \textsc{agn-host} population supports the idea that a merger, having caused a morphological transformation to a smooth galaxy, can also trigger an AGN, causing feedback and cessation of star formation (see Figure 14 of \citealt{Sch2014}).

For the medium mass \textsc{agn-host} population we see a bimodal probability distribution between these two dominant quenching histories, highlighting the strength of this method which is capable of detecting such variation in the SFHs across a population of galaxies. 
\\
\\
We have used morphological classifications from the Galaxy Zoo 2 project to determine the morphology-dependent star formation histories of a population of $1,244$ Type 2 Seyfert AGN host galaxies, in comparison to an inactive galaxy population, via a Bayesian analysis of an exponentially declining star formation history model. We determined the most likely parameters for the quenching onset time, $t_q$, and exponential quenching rate, $\tau$, and find clear differences in the combined population probabilities between inactive and AGN host galaxy populations. We have demonstrated a clear dependence on a galaxy currently hosting an AGN and its SFR. There is strong evidence for downsizing in massive inactive galaxies, which appears as a secondary effect in AGN host galaxies. The dominant quenching mechanism for galaxies currently hosting an AGN is for rapid quenching which has occurred very recently. This result demonstrates the importance of AGN feedback across the entire host galaxy population, in driving the evolution of galaxies across the colour-magnitude diagram.  

%\section*{Acknowledgements}

%RS acknowledges funding from the STFC Grant Code ST/K502236/1. BDS gratefully acknowledges support from Balliol College, Oxford. KS gratefully acknowledges support from Swiss National Science Foundation Grant PP00P2_138979/1. SJK acknowledges funding from the STFC Grant Code ST/MJ0371X/1. KLM acknowledges funding from The Leverhulme Trust as a 2010 Early Career Fellow. KWW acknowledges funding from NSF grant AST-1413610. OIW acknowledges a Super Science Fellowship from the Australian Research Council. The authors would like to thank S. Kaviraj for helpful discussions. The development of GZ was supported by the Alfred P. Sloan Foundation and The Leverhulme Trust. Based on observations made with the NASA GALEX\footnote{\url{http://galex.stsci.edu/GR6/}} and the SDSS\footnote{\url{https://www.sdss3.org/collaboration/boiler-plate.php}}.

%GALEX is operated for NASA by the California Institute of Technology under NASA contract NAS5-98034. Funding for the SDSS and SDSS-II has been provided by the Alfred P. Sloan Foundation, the Participating Institutions, the National Science Foundation, the U.S. Department of Energy, the National Aeronautics and Space Administration, the Japanese Monbukagakusho, the Max Planck Society, and the Higher Education Funding Council for England. %The SDSS Web Site is \url{http://www.sdss.org/}.
%The SDSS is managed by the Astrophysical Research Consortium for the Participating Institutions. The Participating Institutions are the American Museum of Natural History, Astrophysical Institute Potsdam, University of Basel, University of Cambridge, Case Western Reserve University, University of Chicago, Drexel University, Fermilab, the Institute for Advanced Study, the Japan Participation Group, Johns Hopkins University, the Joint Institute for Nuclear Astrophysics, the Kavli Institute for Particle Astrophysics and Cosmology, the Korean Scientist Group, the Chinese Academy of Sciences (LAMOST), Los Alamos National Laboratory, the Max-Planck-Institute for Astronomy (MPIA), the Max-Planck-Institute for Astrophysics (MPA), New Mexico State University, Ohio State University, University of Pittsburgh, University of Portsmouth, Princeton University, the United States Naval Observatory, and the University of Washington.
%This publication made extensive use of the Tool for Operations on Catalogues And Tables (TOPCAT; ~\citealt{Taylor05}). This research has also made use of NASA's ADS service and Cornell's ArXiv. 

\begin{thebibliography}{}

\bibitem[\protect\citeauthoryear{Abazajian et al.}{2009}]{Abazajian09} Abazajian, K. N. et al., 2009, ApJS, 182, 543

\bibitem[\protect\citeauthoryear{Baldry et al.}{2006}]{Baldry06} Baldry, I. et al., 2006, MNRAS, 373, 469

\bibitem[\protect\citeauthoryear{Baldwin, Phillips \& Terlevich}{1981}]{bpt81} Baldwin, J. A., Phillips, M. M., \& Terlevich, R. 1981, PASP, 93, 5

\bibitem[\protect\citeauthoryear{Bamford et al.}{2009}]{Bamford09} Bamford, S. et al., 2009, MNRAS, 393, 1324

\bibitem[\protect\citeauthoryear{Blanton et al.}{2005}]{Blanton05} Blanton, M. R. et al., 2005, AJ, 129, 2562

\bibitem[\protect\citeauthoryear{Blanton \& Roweis}{2007}]{BR07} Blanton, M. R. \& Roweis, S., 2007, AJ, 133, 734

\bibitem[\protect\citeauthoryear{Bower et al.}{2006}]{Bower06} Bower, R. et al., 2006, MNRAS, 370, 645

\bibitem[\protect\citeauthoryear{Brinchmann et al.}{2004}]{Brinch04} Brinchmann, J. et al., 2004, MNRAS, 351, 1151

\bibitem[\protect\citeauthoryear{Bruzual \& Charlot}{2003}]{BC03} Bruzual, G. \& Charlot, S., 2003, MNRAS, 344, 1000

\bibitem[\protect\citeauthoryear{Cardelli et al.}{1989}]{Cardelli89} Cardelli, J. A. et al., 1989, ApJ, 345, 245

\bibitem[\protect\citeauthoryear{Cisternas et al.}{2011}]{Cisternas11} Cisternas, M. et al., 2011, ApJ, 726, 57

\bibitem[\protect\citeauthoryear{Chabrier et al.}{2003}]{Chab03} Chabrier, G., 2003, PASP, 115, 763

\bibitem[\protect\citeauthoryear{Conselice et al.}{2003}]{Con03} Conselice, C. J. et al., 2003, AJ, 126, 1183

\bibitem[\protect\citeauthoryear{Cowie et al.}{1996}]{Cowie96} Cowie, L. et al., 1996, AJ, 112, 839

\bibitem[\protect\citeauthoryear{Cowie \& Barger}{2008}]{CB08} Cowie, L. \& Barger, A. J., 2008, ApJ, 686, 72

\bibitem[\protect\citeauthoryear{Croton et al.}{2006}]{Croton06} Croton, D. J. et al., 2006, MNRAS, 365, 11

\bibitem[\protect\citeauthoryear{Emsellem et al.}{2015}]{Em15} Emsellem, E. et al. 2015, MNRAS, 446, 2468

\bibitem[\protect\citeauthoryear{Fabian}{2006}]{Fab06} Fabian, A. C. 2006, ARA\&A, 50, 455

\bibitem[\protect\citeauthoryear{Foreman-Mackey et al.}{2013}]{Dan} Foreman-Mackey, D., Hogg, D. W., Lang, D., Goodman, J., 2013, PASP, 125, 306

\bibitem[\protect\citeauthoryear{Haines et al.}{2015}]{Haines15} Haines, T. et al., 2015, arXiv:1505.01493

\bibitem[\protect\citeauthoryear{Haring \& Rix}{2004}]{HR04} Haring, N. \& Rix, H-W., 2004, ApJ, 604, 89

\bibitem[\protect\citeauthoryear{Hayward et al.}{2014}]{Hayward14} Hayward, C. C. et al., MNRAS, 442, 1992

\bibitem[\protect\citeauthoryear{Hickox et al.}{2009}]{Hickox09} Hickox, R. C., et al., 2009, ApJ, 696, 891

\bibitem[\protect\citeauthoryear{Hopkins et al.}{2008}]{Hopkins08} Hopkins, F. et al., 2008, ApJSS, 175, 390

\bibitem[\protect\citeauthoryear{Ishibashi et al.}{2012}]{Ish12} Ishibashi, W. et al., 2012, MNRAS, 427, 2998

\bibitem[\protect\citeauthoryear{Kauffman et al.}{2003a}]{Kauff03} Kauffman, G. et al., 2003, MNRAS, 341, 33

\bibitem[\protect\citeauthoryear{Kauffman et al.}{2003b}]{Kauff03b} Kauffman, G. et al., 2003, MNRAS, 346, 1055


\bibitem[\protect\citeauthoryear{Kaviraj et al.}{2013}]{Kav13} Kaviraj, S. et al., 2013, MNRAS, 428, 925

\bibitem[\protect\citeauthoryear{Kaviraj et al.}{2014}]{Kav14} Kaviraj, S. et al., 2014, MNRAS, 440, 2944

\bibitem[\protect\citeauthoryear{Kennicutt}{1997}]{Kennicutt97} Kennicutt, R. C., 1997, ApJ, 498, 491

\bibitem[\protect\citeauthoryear{Kewley et al.}{2001}]{Kew01} Kewley, L. J. et al., 2001, ApJ, 556, 121

\bibitem[\protect\citeauthoryear{Kewley et al.}{2006}]{Kew06} Kewley, L. J. et al., 2006, MNRAS, 372, 961


\bibitem[\protect\citeauthoryear{Kormendy \& Kennicutt}{2004}]{KK04} Kormendy, J. \& Kennicutt, R. J., 2004, ARA\&A, 42, 603

\bibitem[\protect\citeauthoryear{Lintott et al.}{2011}]{Lintott11} Lintott, C. J. et al., 2011, MNRAS, 410, 166

\bibitem[\protect\citeauthoryear{Magorrian et al.}{1998}]{Mag98} Magorrian, J. et al., 1998, AJ, 115, 2285

\bibitem[\protect\citeauthoryear{Marconi \& Hunt}{2003}]{MH03} Marconi, A. \& Hunt, L. K., 2003, ApJ, 589, 21

\bibitem[\protect\citeauthoryear{Martin et al.}{2007}]{Martin07} Martin, D. C. et al., 2007, ApJSS, 173, 342

\bibitem[\protect\citeauthoryear{McIntosh et al.}{2014}]{McIntosh14} McIntosh, D. et al., 2014, MNRAS, 442, 533

\bibitem[\protect\citeauthoryear{Oh et al.}{2011}]{Oh11} Oh, K. et al., 2011, ApJS, 195, 13

\bibitem[\protect\citeauthoryear{Oh et al.}{2015}]{Oh15} Oh, K. et al. 2015, arXiv: 1504.07247

\bibitem[\protect\citeauthoryear{Padmanabhan et al.}{2008}]{Pad08} Padmanabhan, N. et al., 2008, ApJ, 674, 1217

\bibitem[\protect\citeauthoryear{Sarzi et al.}{2010}]{Sarzi10} Sarzi, M. et al., 2010, MNRAS, 402, 2187

\bibitem[\protect\citeauthoryear{Schawinski et al.}{2010}]{Sch2010} Schawinski, K. et al., 2010, MNRAS, 711, 284

\bibitem[\protect\citeauthoryear{Schawinski et al.}{2014}]{Sch2014} Schawinski, K. et al., 2014, MNRAS, 440, 889

\bibitem[\protect\citeauthoryear{Schmidt}{1959}]{Schmidt59} Schmidt, M., 1959, ApJ, 129, 243

\bibitem[\protect\citeauthoryear{Silk \& Rees}{1998}]{SR98} Silk, J. \& Rees, M. J., 1998, A\&A, 331, L1

\bibitem[\protect\citeauthoryear{Simmons et al.}{2011}]{Simmons11} Simmons, B. D. et al., 2011, ApJ, 734, 121

\bibitem[\protect\citeauthoryear{Singh et al.}{2013}]{Singh13} Singh, R. et al., 2013, A\&A, 558, 43

\bibitem[\protect\citeauthoryear{Snyder et al.}{2011}]{Snyder11} Snyder, G. F. et al., 2011, ApJ, 741, 77

\bibitem[\protect\citeauthoryear{Smethurst et al.}{2015}]{Sme2015} Smethurst, R. J. et al., 2015, MNRAS, 450, 435

\bibitem[\protect\citeauthoryear{Somerville et al.}{2008}]{Somer08} Somerville, R. S. et al., 2008, MNRAS, 391, 481

\bibitem[\protect\citeauthoryear{Springel, Di Matteo \& Hernquist}{2005}]{SdMH05} Springel, V., Di Matteo, T. \& Hernquist, L., 2005, ApJ, 620, L79

\bibitem[\protect\citeauthoryear{Taylor}{2005}]{Taylor05} Taylor, M. B., 2005, ASP Conference Series, 347

\bibitem[\protect\citeauthoryear{Thomas et al.}{2010}]{Thomas10} Thomas, D. et al., 2010, MNRAS, 404, 1775

\bibitem[\protect\citeauthoryear{Tortora et al.}{2009}]{Torbra09} Tortora, C. et al., 2009, MNRAS, 369, 61

\bibitem[\protect\citeauthoryear{Varela et al.}{2004}]{Varela04} Varela, J. et al., 2004, A\&A, 420, 873

\bibitem[\protect\citeauthoryear{Wild et al.}{2009}]{Wild09} Wild, V. et al., 2009, MNRAS, 395, 144

\bibitem[\protect\citeauthoryear{Willett et al.}{2013}]{GZ2} Willett, K. et al., 2013, MNRAS, 435, 2835

\bibitem[\protect\citeauthoryear{Yan \& Blanton}{2012}]{RB12} Yan, R. \& Blanton, M. R. 2012, ApJ, 747, 61

\bibitem[\protect\citeauthoryear{Yesuf et al.}{2014}]{Yesuf14} Yesuf, H. M. et al., 2014, ApJ, 792, 84

\bibitem[\protect\citeauthoryear{York et al.}{2000}]{York2000} York, D. G. et al., 2000, AJ, 120, 1579

\bibitem[\protect\citeauthoryear{Zinn et al.}{2013}]{Zinn13} Zinn, P. et al., 2013, ApJ, 774, 66

\end{thebibliography}{}

\appendix

\section{Mass matched INACTIVE sample}


\begin{figure}
\includegraphics[width=0.48\textwidth]{bpt/mass_distributions_low_med_high_mass_inactive_sample_mass_pd_ps_matched.png}
\caption{Histograms of the distribution of stellar masses in the \textsc{agn-host} (solid) and matched \textsc{inactive} (dashed) samples, split into low (left), medium (middle) and high (right) mass bins. A Kolmogorov-Smirnov test reveals that the two stellar mass distributions are statistically indistinguishable $(D \sim 0.01, p \sim 0.99)$.}
\label{massm}
\end{figure}

\begin{figure}
\includegraphics[width=0.48\textwidth]{bpt/pd_vote_fraction_distribution_low_med_high_mass_inactive_sample_mass_pd_ps_matched.png}
\caption{Histograms of the distribution of GZ disc vote fractions, $p_d$, in the \textsc{agn-host} (solid) and matched \textsc{inactive} (dashed) samples, split into low (left), medium (middle) and high (right) mass bins. A Kolmogorov-Smirnov test reveals that the two $p_d$ distributions are statistically indistinguishable $(D \sim 0.03, p \sim 0.39)$.}
\label{pdm}
\end{figure}

\begin{figure}
\includegraphics[width=0.48\textwidth]{bpt/ps_vote_fraction_distribution_low_med_high_mass_inactive_sample_mass_pd_ps_matched.png}
\caption{Histograms of the distribution of GZ smooth vote fractions, $p_s$, in the \textsc{agn-host} (solid) and matched \textsc{inactive} (dashed) samples, split into low (left), medium (middle) and high (right) mass bins. A Kolmogorov-Smirnov test reveals that the two $p_s$ distributions are statistically indistinguishable $(D \sim 0.04, p \sim 0.05)$.}
\label{psm}
\end{figure}

Each galaxy in the \textsc{agn-host} sample has been matched to at least one and upto five inactive galaxies. These were matched to within $\pm5\%$ of the stellar mass and $\pm 0.1$ of each of the disc and smooth GZ vote fractions, $p_d$ and $p_s$. The distributions of these three variables are shown for both the \textsc{agn-host} and \textsc{inactive} samples in the three mass bins in Figures \ref{massm}, \ref{pdm} \& \ref{psm}. 

Both the \textsc{agn-host} and \textsc{inactive} galaxy samples are shown on an optical-NUV colour colour diagram in Figure \ref{colcol}. The \textsc{inactive} sample across all mass bins can be seen to encompass the entirety of the colour magnitude diagram, unlike the \textsc{agn-host} sample which reside at increasingly green colours with increasing mass. 


\begin{figure*}
\includegraphics[width=0.8\textwidth]{bpt/colour_colour_matched_sample_inactive_seyferts.png}
\caption{Optical-NUV colour-colour contour diagrams for the \textsc{agn-host} (top) and \textsc{inactive} galaxy samples split into low (blue), medium (green) and high (red) stellar mass samples. Underlaying each diagram are the contours of the \textsc{gz2-galex} sample (grey).}
\label{colcol}
\end{figure*}


\section{Luminosity dependence}

An investigation into the dependence of the quenching in the \textsc{agn-host} sample with $L[OIII]$ was also conducted, with the summed weighted posterior probability distributions for the quenching time and rate parameters shown in Figures \ref{loiiitime} \& \ref{loiiirate}. The \textsc{agn-host} sample was split into low, medium and high luminosity as with the stellar mass. Since the  $L[OIII]$ is dependent on the accretion rate, which is correlated with the mass of the black hole, which is in turn correlated with the stellar mass of the host galaxy, the stellar mass was used in the main investigation to allow a direct comparison to the control \textsc{inactive} sample. 


\begin{figure}
\includegraphics[width=0.35\textwidth]{bpt/quenching_time_histograms_smooth_red_disc_blue_vertical_luminosity_bpt_seyf_only_hardcut_minimal.png}
\caption{Likehood distribution for the quenching time, $t_q$ normalised so that the areas under the curves are equal. \textsc{agn-host} galaxies are split into low (top), medium (middle) and high (bottom) $L[OIII]$ for smooth (red dashed) and disc (blue solid) galaxies. A low value of $t_q$ corresponds to the early Universe and a high value to the recent Universe.}
\label{loiiitime}
\end{figure}

\begin{figure}
\includegraphics[width=0.35
\textwidth]{bpt/quenching_rate_histograms_smooth_red_disc_blue_vertical_luminosity_bpt_seyf_only_hardcut_minimal.png}
\caption{Likehood distribution for the quenching rate, $\tau$ normalised so that the areas under the curves are equal. \textsc{agn-host} galaxies are split into low (top), medium (middle) and high (bottom) $L[OIII]$ for smooth (red dashed) and disc (blue solid) galaxies. A small (large) value of $\tau$ corresponds to a rapid (slow) quench.}
\label{loiiirate}
\end{figure}


\end{document}
